\documentclass{amsart}

\usepackage[utf8]{inputenc}
\usepackage{lmodern}
\usepackage[T1]{fontenc}
\usepackage{amsmath,amssymb,amsthm,enumerate}
\usepackage{mathpartir}
\usepackage{tikz}
\usepackage{tikz-cd}
\usepackage{url}


\theoremstyle{plain}
\newtheorem{theorem}{Theorem}
\newtheorem*{theorem*}{Theorem}
\newtheorem{lemma}[theorem]{Lemma}
\newtheorem*{lemma*}{Lemma}
\newtheorem{corollary}[theorem]{Corollary}
\newtheorem*{corollary*}{Corollary}
\newtheorem{proposition}[theorem]{Proposition}
\newtheorem*{proposition*}{Proposition}
\newtheorem{claim}[theorem]{Claim}
\newtheorem*{claim*}{Claim}
\newtheorem{conjecture}[theorem]{Conjecture}
\newtheorem*{conjecture*}{Conjecture}
\newtheorem*{namedtheorem}{\theoremname}
\newenvironment{named}[1]
{\renewcommand{\theoremname}{#1}\begin{namedtheorem}}
  {\end{namedtheorem}}

\theoremstyle{definition}
\newtheorem{definition}[theorem]{Definition}
\newtheorem*{definition*}{Definition}
\newtheorem{notation}[theorem]{Notation}
\newtheorem*{notation*}{Notation}
\newtheorem{example}[theorem]{Example}
\newtheorem*{example*}{Example}
\newtheorem{examples}[theorem]{Examples}
\newtheorem*{examples*}{Examples}

\theoremstyle{remark}
\newtheorem{remark}[theorem]{Remark}
\newtheorem*{remark*}{Remark}
\newtheorem{note}[theorem]{Note}
\newtheorem*{note*}{Note}
\newtheorem*{acknowledgments*}{Acknowledgments}

\newcommand{\der}{\vdash}
\newcommand{\co}{\colon}
\newcommand{\op}{\mathrm{op}}

\newcommand{\C}{\mathcal{C}}
\newcommand{\D}{\mathcal{D}}
\newcommand{\T}{\mathcal{T}}
\newcommand{\ESet}{\mathbf{ESet}}

\DeclareMathOperator{\Ty}{Ty}
\DeclareMathOperator{\Ter}{Ter}

\newcommand{\emptyctx}{\mathbf{1}}
\newcommand{\emptysubst}{\mathbf{!}}
\newcommand{\pp}{\mathsf{p}}
\newcommand{\qq}{\mathsf{q}}



\title{Towards an initial E-cwf with a total interpretation function
  in Agda}
\date{\today}
\author{Peter Dybjer and Simon Huber}

\begin{document}
\maketitle

\section{Introduction}
\label{sec:introduction}

We are in the progress of formalizing a notion of initial cwf in type
theory using Agda, not assuming any extensionality axioms, no quotient
types etc.  Instead we are basing the development on setoids and
E-cwfs.  We are defining an E-category (not yet formalized) of E-cwfs
and weak E-cwf morphisms.  In this category we want to construct an
initial E-cwf as follows:
\begin{itemize}
\item We define a type of \emph{raw terms} for cwf-combinators (i.e.,
  explicit substitutions).
\item We then mutually inductively define seven relations
  corresponding to the different forms of judgments.  Note that we do
  not have a judgment for context equality.  Each introduction rule
  for these relations corresponds to an inference rule for a version
  of type theory based on cwf-combinators.
\item We prove that these can be organized into an E-cwf.
\end{itemize}
We then construct an E-cwf morphism (i.e., an interpretation function)
from the above E-cwf into a given E-cwf.  Unlike some related work,
this is not constructed as a partial function on the raw syntax.
Instead it is constructed by induction on derivations of judgments,
with on case for each inference rule.  The uniqueness of this E-cwf
morphism (up to natural isomorphism) has not yet been established.

The code referred to in this note can be found at
\url{https://github.com/simhu/ecwf/}.

\subsection{Related work}
\label{sec:related-work}

\begin{itemize}
\item Brunerie and de~Boer,
  \url{https://github.com/guillaumebrunerie/initiality}, partial
  interpretation function based on proof-irrelevant Prop in Agda,
  assuming quotients and function extensionality.
\item Lumsdaine and Mörtberg, in progress, partial interpretation
  function, based on UniMath
  \url{https://github.com/UniMath/TypeTheory} on branch
  \texttt{initiality-add-nat}
\item Altenkirch and Kaposi, based on QIITs
\end{itemize}

\section{E-categories with families}
\label{sec:e-cwfs}

We define setoids, E-categories, and E-functors as usual.  The
E-category of E-sets (i.e., setoids) $\ESet$ is defined
usual.\footnote{\texttt{src/Basics.agda}} Given a E-presheaf $F \co
\C^\op \to \ESet$ (i.e., a contravariant E-functor into $\ESet$) we
define the \emph{E-category of elements} $\int F$ as one
expects.\footnote{\texttt{src/Presheaves.agda}}

\begin{definition}
  An \emph{E-cwf}\footnote{\texttt{src/Cwf/Elem.agda}; another (start
    of a) definition based on E-families is in
    \texttt{src/Cwf/Fam.agda}} is given by:
  \begin{itemize}
  \item an E-category $\C$ of \emph{contexts} and
    \emph{substitutions};
  \item an E-presheaf $\Ty$ of \emph{types} on $\C$;
  \item an E-presheaf $\Ter$ of \emph{terms} on $\int \Ty$;
  \item an terminal object $\emptyctx$ of $\C$ with morphisms
    $\emptysubst_\Gamma \co \Gamma \to \emptyctx$;
  \item context comprehension of $\Gamma$ in $\C$ and $A$ in $\Ty
    (\Gamma)$, formulated as terminal object in the E-category of
    triples $\Delta,\sigma, t$.
  \end{itemize}
\end{definition}

\begin{remark}
  Part of this definition are ``transport'' operations: given $p : A
  \sim B$ in the setoid $\Ty(\Gamma)$, we get a map $\iota\,p : \Ter
  (\Gamma, B) \to \Ter(\Gamma,A)$, which is \emph{proof-irrelevant} in
  the sense that if also $q : A \sim B$ and $u \sim v$ in $\Ter
  (\Gamma,B)$, then $\iota\,p\,u \sim \iota\,q\,v$ in $\Ter
  (\Gamma,A)$.  This is a consequence of the definition of the
  E-category of elements.
\end{remark}

\begin{definition}
  An \emph{morphism of E-cwf}\footnote{\texttt{src/Cwf/Elem.agda}} $F$
  between E-cwfs $\C$ and $\D$ is given by:
  \begin{itemize}
  \item an E-functor $F$ on underlying categories of contexts;
  \item an E-natural transformation $F$ between $\Ty_\C$ and $\Ty_D
    \circ F^\op$;
  \item an E-natural transformation between $\Ter_\C$ and $\Ter_\D
    \circ \int F$;
  \item the induced maps
    \[
      F \emptyctx_\C \to \emptyctx_\D%
      \qquad \text{and } \qquad%
      F (\Gamma. A) \to F \Gamma. F A
    \]
    are isomorphisms.
  \end{itemize}
\end{definition}

Note that we can not define the notion of \emph{strict} E-cwf morphism
because we have no natural notion of context equality.

\begin{remark}
  In this framework, the natural choice of equivalence relation
  between E-functors is E-natural isomorphism.  With this choice one
  can organize E-categories themselves into an E-category.  Similarly,
  we can organize E-cwfs into an E-category.
\end{remark}

\section{Term model}
\label{sec:term-model}

We begin by inductively defining the type of \emph{raw} expressions,
encompassing raw terms, types, contexts, and explicit substitutions.
Note, we have certain annotations, e.g., on composition with the
mediating object, substitution on types with the codomain, and
substitution on terms with type and codomain.  But we don't have
annotations on identities of projections.

We then mutually inductively define the types of derivations of
judgments (as indexed families).  The seven forms of judgments are:
\begin{mathpar}
  \Gamma \der \and%
  \Gamma \der A \and%
  \Gamma \der A \sim B \and%
  \Gamma \der u \in A \and%
  \Gamma \der u \sim v \in A \and%
  \sigma \in \Delta \Rightarrow \Gamma \and%
  \sigma \sim \tau \in \Delta \Rightarrow \Gamma \and%
\end{mathpar}

We have experimented with several versions, the current one found in
\texttt{src/Syntax/ExplicitExtrinsicConvNf.agda}.  What is special
about this current approach is that we drop the type-conversion rule
and replace by \emph{local} applications of the type-conversion rule.
In this way derivations are syntax directed, in the sense that
for each term constructor there is \emph{only one} typing rule:
\begin{mathpar}
  \inferrule %
  { \Gamma \der A \\ \Gamma.A \der A [ \pp ] \sim B }%
  { \Gamma . A \der \qq \in B} %
  \and %
  \inferrule %
  { \Gamma \der \\
    \Gamma \der t \in A \\
    \sigma \in \Delta \Rightarrow \Gamma \\
    \Gamma \der A\\
    \Delta \der A [ \sigma ]_{\Gamma} \sim B
  } %
  { \Delta \der t [ \sigma ]_{\Gamma, A} \in B }
\end{mathpar}
This will be essential when defining the total interpretation
function.

\begin{remark}
  An alternative could be to follow Curien and add an explicit term
  constructor for the general type-conversion
  rule:\footnote{\texttt{src/Syntax/ExplicitExtrinsicCoe.agda}}
  \[
    \inferrule {\Gamma \der t \in A \\ \Gamma \der B \sim A}
    {\Gamma \der \iota_{A,B}\,t \in B}
  \]
\end{remark}

\begin{remark}
  There are several choices one could take while building a term
  model: explicit versus implicit substitutions, the annotations on
  raw terms, the premises in rules.\footnote{\texttt{src/Syntax/}} The
  current setup was refined while trying to define the interpretation
  function.
\end{remark}

\begin{theorem}
  The term model defines an E-cwf $\T$.\footnote{See \texttt{SynCwf}
    in \texttt{src/Syntax/ExplicitExtrinsicConvNf.agda}.}
\end{theorem}


\section{The interpretation function}
\label{sec:interpretation}

Given a E-cwf $\C$ we have to define a morphism $\T \to \C$.  To do
so, we define several components
simultaneously.\footnote{\texttt{src/Syntax/ExplicitExtrinsic/Initial.agda}}
The intention is that these functions are defined by induction on the
derivations.  The idea is that the given interpretation of an
inference rule is a function from the interpretation of the premises
to the interpretation of the conclusion.  Although, one would expect
this definition to be structurally-recursive, it is not accepted by
the Agda termination checker, and it is not even clear how to give an
informal termination argument based on a measure on the complexity of
derivations.

While defining the interpretation functions we need that our
interpretations are \emph{independent} of the given derivations.  Two
derivations $p\sigma, p\sigma'$ of a judgment $\sigma \in \Delta
\Rightarrow \Gamma$ should give rise to two related substitutions in
$\C$.  As there is no primitive notion to compare two contexts in $\C$
we formulate the independence for contexts judgments using
isomorphisms: two derivations $p\Gamma, p\Gamma'$ of $\Gamma \der$
should give rise to isomorphic contexts in $\C$.  And these
isomorphism should harmonize suitably with rest of the interpretation.

\begin{remark}
  The fact that all judgments defined simultaneously is a feature of
  dependent type theory.  This phenomena does not appear when we,
  e.g., define the free E-category given a
  graph\footnote{\texttt{src/FreeCat.agda}} of free E-category with
  products\footnote{\texttt{src/Products.agda}} where objects are
  separate from morphisms.
\end{remark}

\begin{itemize}
\item An E-functor on the underlying categories of contexts, which
  again has several components: we give
  \begin{itemize}
  \item a function on objects $\mathtt{o}$ and auxiliary functions
    \begin{itemize}
    \item a function that maps two derivations $p\Gamma, p\Gamma'$ of
      $\Gamma \der$ to a morphism between $\mathtt{o} p\Gamma$ and
      $\mathtt{o} p\Gamma'$, and
    \item witnesses $\mathtt{o\# id}, \mathtt{o\# comp}$ that
      $\mathtt{o\#}$ is suitably functorial.
    \end{itemize}


  \item a function on morphisms $\mathtt{m}$ and auxiliary functions:
    \begin{itemize}
    \item a function $\mathtt{m\#}$ stating independence of
      $\mathtt{m}$, and
    \item $\mathtt{m-o\#}$ witnessing how $\mathtt{m}$ interacts with
    $\mathtt{o\#}$.
    \end{itemize}
  \item a proof $\mathtt{m-resp}$ that $\mathtt{m}$ preserves the
    equivalence relation
  \item proofs that $\mathtt{m}$ preserves identity and composition;
    for the former we need $\mathtt{o\# id}$, the latter is by
    definition.
  \end{itemize}
\item An E-natural transformation for types, consisting of:
  \begin{itemize}
  \item A function $\mathtt{ty}$ assigning types in $\C$ to
    derivations of types, with auxiliary functions:
    \begin{itemize}
    \item $\mathtt{ty\#}$ stating independence of type derivations,
      together with independence of the underlying context
    \end{itemize}
  \item A function $\mathtt{ty-cong}$ that $\mathtt{ty}$ preserves the
    setoid relation.
  \item A witness for naturality (substitution on types).
  \end{itemize}

\item An E-natural transformation for terms, consisting of:
  \begin{itemize}
  \item A function $\mathtt{ter}$ assigning terms in $\C$ to
    derivations of terms, with auxiliary functions:
    \begin{itemize}
    \item $\mathtt{ter\#-ty-eq}$ stating independence of term
      derivations, together with independence of the underlying types
      and context (where we allow two judgmentally equal types).
    \end{itemize}
  \item A function $\mathtt{ter-cong}$ that $\mathtt{ter}$ preserves
    the setoid relation.
  \item A witness for naturality.
  \end{itemize}

\item The proofs that terminal object and context comprehension are
  preserved, should be direct by how the definition is set up (but is
  not formalized yet).
\end{itemize}


\subsection{Agda type checking}

Agda type checker accepts the definition when turning off the
termination checker.  The definition of the derivations is an indexed
inductive family in the usual sense with a definite elimination rule.
However, as it is written our interpretation function is not expressed
in terms of this eliminator.  Instead we make use of Agda's powerful
way of doing dependently typed pattern matching, often matching on
several different arguments simultaneously.  Moreover, when Agda type
checks mutual recursive definitions the right-hand sides of our
equations are checked in order, and when checking a given right-hand
side the definitional equalities of the previous functions are
available.


\subsection{Termination}

Most of the recursive calls are on clearly structurally smaller
arguments.  But there are several exceptions which are probably the
cause of Agda's termination checker not accepting our definition.  (A
particular complex case is the interpretation of substitution
extension $\langle \sigma, t \rangle$ by the function $\mathtt{m}$.)

One possibility would be to assign a measure on the recursive calls,
for example, summing the sizes of the derivations, or using multiset
ordering.  However, neither of these succeeds because of either
duplication or ``reshuffling'' of the arguments.

What does not seem to work is to do an induction on the structure of
raw syntax, since this breaks when interpreting equality of types (in
the case of transitivity).

\subsection{Directions}

Another possibility one could try is to define the interpretation
referring to ``inversion'' functions (e.g., a derivation $\Gamma \der
A$ includes a derivation of $\Gamma \der$).  Then the interpretation
of a type derivation $\Gamma \der A$ would be a type over the
interpretation of the \emph{induced} derivation $\Gamma \der$ (and not
a given derivation of $\Gamma \der$).  A problem with this approach is
that we have to keep track that the induced derivations don't increase
in size.




\end{document}

%%% Local Variables:
%%% mode: latex
%%% TeX-master: t
%%% End:
